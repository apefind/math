
\newpage
\section{The Road to Reality}


\subsection{Hyperbolic Geometry}
The ratio between the area \( A \) and \( A' \) of two similar shapes is given by
\[
	A' = k^2A
\]
\bigskip

\begin{theorem}[Pythagoras]\label{thm:thm_pythagoras}
\[
    a^2 + b^2 = c^2
\]
\end{theorem}


\begin{proof}
Let \( A, B \) and \( C \) be the areas of the three triangles respectively. All triangles are similar, hence
\[
    B = \frac{b^2}{a^2} A \text{ and }  C =  \frac{c^2}{b^2} B 
\]
Since \( A + B = C \) it follows that
\[
     a^2 + b^2 = \frac{b^2A}{B} + b^2 = \frac{b^2(A + B)}{B} = \frac{b^2 C}{B} = c^2
\]
\end{proof}
\bigskip

% \begin{figure}[H]
% \centering
% \begin{tikzpicture}
% \rectriangle[black]{(0,0)}{(3,3)}{6cm}
% \end{tikzpicture}
% \caption{Pythagoras}\label{fig:pythagoras}
% \end{figure}
% \bigskip


\lemma[Conformal and Projective Representation]
The mapping from conformal and projective representation of any point is given by the radial expansion 
of the following factor
\[
   \frac{2R}{R^2 + r^2}
\]  

\begin{proof}
For any point the distance from the origin with regard to the two representations is given by
\[
    \log \frac{R + r}{R - r} = \frac{1}{2} \log \frac{R + r'}{R - r'} = \log \frac{(R + r')^2}{(R - r')^2}
\]
This gives
\[
  {(R - r)}^2 (R + r') = {(R + r)}^2 (R - r') \text{ and } -4R^2r + 2R^2r' + 2r^2r' = 0
\] 
Hence
\[
   r' = \frac{2R^2}{R^2 + r^2} r
\] 
\end{proof}


\subsection{Complex Numbers}

\begin{lemma}[Basic Formulas]\hfill
    \begin{enumerate}
        \item It is
            \[
                (a + \iu b) (c + \iu d) = (ac - bd) + \iu (ad + bc)
            \]
        \item Thus
            \[
               {(a + \iu b)}^2 = (a^2 - b^2) + \iu 2ab
            \]
            and
            \[
                (a + \iu b) (a - \iu b) =  a^2 + \iu ab - \iu ab - \iu^2b^2 = a^2 + b^2
            \]
        \item Hence
            \[
                \frac{a + \iu b}{c + \iu d} = \frac{(a + \iu b)(c - \iu d)}{c^2 + d^2} = 
                    \frac{ac + bd}{c^2 + d^2} + \iu \frac{bc - ad}{c^2 + d^2}
            \]
        \item For 
            \[
                z = \sqrt{\frac{1}{2}(a + \sqrt{a^2 + b^2})} + \iu \sqrt{\frac{1}{2}(-a + \sqrt{a^2 + b^2})} 
            \]
            it follows
            \[
                z^2 = \frac{1}{2}(a + \sqrt{a^2 + b^2}) - \frac{1}{2}(-a + \sqrt{a^2 + b^2}) +
                    \iu 2\sqrt{\frac{1}{4} (\sqrt{a^2 + b^2}^2) - a^2} = a + \iu b
            \]
    \end{enumerate}
\end{lemma}
\bigskip


\begin{lemma}[Binomial Theorem]\hfill
    \begin{enumerate}
        \item For the binomial coefficient Pascal's identity holds
			\[
				\binom{n}{k - 1} + \binom{n}{k} = \binom{n + 1 }{k }   
			\]  
        \item The following equation states the binomial identity
			\[
				{(a + b)}^n = \sum_{k=0}^{n} \binom{n}{k} a^{k} b^{n - k} = \sum_{k=0}^{n} \binom{n}{k} a^{n -k} b^{k}
			\]  
        \item For \( a = 1 \) follows
			\[
				{(1 + x)}^n = \sum_{k=0}^{n} \binom{n}{k} x^{k}
			\]  
    \end{enumerate}
\end{lemma}

\begin{proof}
It is
\[
	\binom{n}{k} + \binom{n}{k - 1} = \frac{n!}{k!(n - k)!} + \frac{n!}{(k - 1)!(n - k + 1)!}
		= \frac{n!(n + 1 - k) + n!k!}{k!(n + 1- k)!} = \binom{n + 1}{k}
\]
Furthermore by using induction
\[
    \begin{split}
		{(a + b)}^{n + 1}	& = \sum_{k=0}^{n} \binom{n}{k} a^{k + 1} b^{n - k} + 
								\sum_{k=0}^{n} \binom{n}{k} a^{k} b^{n + 1 - k} \\
							& = \sum_{k=1}^{n + 1} \binom{n}{k - 1} a^{k} b^{n + 1 - k} + 
								\sum_{k=0}^{n} \binom{n}{k} a^{k} b^{n + 1 - k} \\
							& = \sum_{k=0}^{n + 1} \binom{n + 1}{k} a^{k} b^{n + 1 - k}
    \end{split}
\]  
\end{proof}
\bigskip


\subsection{Exponential Function and Logarithms}

\begin{exercise}[Exponential Function]
The Cauchy product yields
\[
	\sum_{n=0}^\infty a_n \sum_{n=0}^\infty b_n = \sum_{n=0}^\infty \sum_{k=0}^n a_k b_{n-k}
\]  
if at least one of the series is absolutely convergent. Hence
\[
    \begin{split}
		\sum_{n=0}^\infty \frac{1}{n!} z^n \sum_{n=0}^\infty \frac{1}{n!} w^n 
			& = \sum_{n=0}^{\infty} \sum_{k=0}^{n} \frac{1}{k!} z^{k} \frac{1}{(n - k)!} w^{n - k} \\
			& = \sum_{n=0}^{\infty} \frac{1}{n!} \sum_{k=0}^{n} \binom{n}{k} z^{k}w^{n - k} \\
			& = \sum_{n=0}^\infty \frac{1}{n!} {(z + w)}^n	
    \end{split}
\]  
\end{exercise}
\bigskip


Let \( t \in \R \). Then
\[
    \begin{split}
		\eu^{\iu t}
			& = \sum_{k=0}^\infty \frac{1}{k!} {(\iu t)}^k \\
			& = \sum_{k=0}^\infty \frac{1}{2k!} {(\iu t)}^{2k} + 
				\sum_{k=0}^\infty \frac{1}{(2k + 1)!} {(\iu t)}^{2k + 1} \\
			& = \sum_{k=0}^\infty \frac{(-1)^k}{2k!} t^{2k} + 
				\iu\sum_{k=0}^\infty \frac{(-1)^k}{(2k + 1)!} t^{2k + 1} \\
			& = \cos{t} + \iu\sin{t}
    \end{split}
\]

More generally for \( z = \log r + \iu t\)
\[
	\eu^z = \eu^{\log r + \iu t} = r\eu^{\iu t} = r(\cos{t} + \iu\sin{t})
\]

For \( r = 1 \) and \( t = 2\pi \) this yields
\[
	\eu^{2\pi\iu} = \cos{2\pi} + \iu \sin{2\pi} = 1
\]
and for \( t = 2\pi \) we get
\bigskip

\begin{lemma}[Euler Equation]\label{lemma:lemma_euler_equation}
\[
	\eu^{\pi\iu} + 1 = 0
\]
\end{lemma}
\bigskip


\begin{exercise}\hfill
    \begin{enumerate}
        \item If \( \eu^z = w \) then \( z + \pi\iu \) is a logarithm to \( -w \):
			\( \eu^{z + \pi\iu} = \eu^{z}\eu^{\pi\iu} = - \eu^{z} = -w \).
		\item Since \( \eu^{i(s + t)} = \eu^{is} \eu^{it} \) it follows
			\[
				\begin{split}
					\cos{(s + t)} + \iu \sin{(s + t)} 
						& = (\cos{s} + \iu\sin{s})(\cos{t} + \iu\sin{t}) \\
						& = \cos{s}\cos{t} - \sin{s}\sin{t} + \iu(\cos{s}\sin{t} + \sin{s}\cos{t})
				\end{split}
			\]
			Hence
			\[
				\begin{split}
					\cos{(s + t)} & = \cos{s}\cos{t} - \sin{s}\sin{t} \\
					\sin{(s + t)} & = \cos{s}\sin{t} + \sin{s}\cos{t}
				\end{split}
			\]
		\item It is \( \eu^{3it} = (\eu^{it})^3 \) and thus
			\[
				\cos{3t} + \iu \sin{3t} 
					= (\cos{t} + \iu\sin{t})^3
					= \cos^3{t} - 3\cos{t}\sin^2(t) + \iu(\cos^2{t}\sin{t} -\sin^3{t})
			\]
        \item Fun facts
			\[ 
				\eu^{1 - 4\pi^2} = \eu^{1 + (2\iu\pi)^2} = \eu \eu^{2\pi\iu}\eu^{2\pi\iu} = \eu
			\]
			and \( \iu = \eu^{\iu\pi/2} \) gives
			\[
				\iu^\iu = \eu^{\iu\log\iu} = \eu^{\iu\iu\pi/2} = \eu^{-\pi/2} \in \R
			\]
    \end{enumerate}
\end{exercise}
\bigskip


\subsection{Complex Number Calculus}


\begin{theorem}[Cauchy Riemann Equations]\label{thm:thm_cauchy_riemann_eqations}
Let \( f = u + \iu v \) be holomorphic. Then \( f \) satisfies the Cauchy Riemann equations
\[
		\frac{\partial u}{\partial x} = \frac{\partial v}{\partial y}
\]
\[
		\frac{\partial u}{\partial y} = - \frac{\partial v}{\partial x}
\]
\end{theorem}

\begin{proof}
For \( h \in \R \) follows
\[
	\lim_{h \to 0}\frac{f(z + h) - f(z)}{h} = \frac{\partial u}{\partial x}(z) + \iu \frac{\partial v}{\partial x}(z)
\]	
and
\[
	\lim_{h \to 0}\frac{f(z + \iu h) - f(z)}{\iu h}
		= \frac{\partial u}{\iu\partial y}(z) + \frac{\partial v}{\partial y}(z)
		= \frac{\partial v}{\partial y}(z) - \iu\frac{\partial u}{\partial y}(z)
\]	

\end{proof}
\bigskip


\begin{exercise}\hfill
    \begin{enumerate}
        \item Let \( f(z) = z^3 \). Then \( u(x,y) + \iu v(x,y) = x^3 - 3xy^2 + \iu (3x^2y -y^3) \) and as expected
			\[
				\frac{\partial u}{\partial x}(x,y) = x^3 - 3y^2 \quad 
				\frac{\partial u}{\partial y}(x,y) = -6xy \quad\text{ and }\quad
				\frac{\partial v}{\partial x}(x,y) = 6xy \quad
				\frac{\partial v}{\partial y}(x,y) = x^3 - 3y^2
			\]
        \item Let \( z_0 \in \C \) and \( \gamma(t) = z_0 + \eu^{\iu t} \) for \( t \in [0, 2\pi] \). Then
			\[
				\int_\gamma \frac{1}{z - z_0}\,dz 
					= \int_0^{2\pi} \frac{\iu \eu^{\iu t}}{z_0 + \eu^{\iu t} - z_0}\,dt
					= \int_0^{2\pi} \iu\,dt
					= 2\pi\iu
			\]
			and thus \( 1/ (z - z_0 ) \) has no anti-derivative on \( \C \setminus \{z_0\} \)
    \end{enumerate}
\end{exercise}
\bigskip
