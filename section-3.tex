
\newpage
\section{The Road to Reality}


\subsection{Hyperbolic Geometry}

\begin{theorem}[Pythagoras]\label{thm:thm_pythagoras}
\[
    a^2 + b^2 = c^2
\]
\end{theorem}


\begin{proof}
All triangles are similar, hence
\[
    B = \frac{b^2}{a^2} A \text{ and }  C =  \frac{c^2}{b^2} B 
\]
Since \( A + B = C \) it follows that
\[
     a^2 + b^2 = \frac{b^2A}{B} + b^2 = \frac{b^2(A + B)}{B} = \frac{b^2 C}{B} = c^2
\]
\end{proof}
\bigskip

\begin{figure}[H]
\centering
\begin{tikzpicture}

\coordinate[label={150:$A$}] (A) at (135:4);
\coordinate[label={270:$C$}] (C) at (0:0);
\coordinate[label={30:$B$}] (B) at (45:5);
\draw (C) -- (A)node[midway,below left]{$b$} -- (B)  -- cycle node[midway,below right]{$a$};
\draw (C) -- ($(A)!(C)!(B)$) coordinate (P) node[midway,right]{$h$};
\draw[decorate,decoration={brace,raise=12pt,amplitude=5pt}] (A) -- (B);
\path (B) -- (P) node[midway,above]{$x$} -- (A) node[midway,above]{$y$} ($($(A)!0.5!(B)$)!0.8cm!90:(B)$) node{$C$};
\filldraw[fill=white] (C) -- ($(C)!2mm!(A)$) coordinate (U) -- ($(U)!2mm!90:(C)$) --($(C)!2mm!(B)$) --cycle;
\draw ($(P)!2mm!(C)$) coordinate (V) -- ($(V)!2mm!90:(C)$) --($(P)!2mm!(B)$);

\end{tikzpicture}
\caption{Pythagoras}\label{fig:pythagoras} 
\end{figure}
\bigskip


\lemma[Conformal and Projective Representation]
The mapping from conformal and projective representation of any point is given by the radial expansion 
of the following factor
\[
   \frac{2R}{R^2 + r^2}
\]  

\begin{proof}
For any point the distance from the origin with regard to the two representations is given by
\[
    \log \frac{R + r}{R - r} = \frac{1}{2} \log \frac{R + r'}{R - r'} = \log \frac{(R + r')^2}{(R - r')^2}
\]
This gives
\[
  {(R - r)}^2 (R + r') = {(R + r)}^2 (R - r') \text{ and } -4R^2r + 2R^2r' + 2r^2r' = 0
\] 
Hence
\[
   r' = \frac{2R^2}{R^2 + r^2} r
\] 
\end{proof}


\subsection{Complex Numbers}

\begin{lemma}[Basic Formulas]\hfill
    \begin{enumerate}
        \item It is
            \[
                (a + \iu b) (c + \iu d) = (ac - bd) + \iu (ad + bc)
            \]
        \item Thus
            \[
               {(a + \iu b)}^2 = (a^2 - b^2) + \iu 2ab
            \]
            and
            \[
                (a + \iu b) (a - \iu b) =  a^2 + \iu ab - \iu ab - \iu^2b^2 = a^2 + b^2
            \]
        \item Hence
            \[
                \frac{a + \iu b}{c + \iu d} = \frac{(a + \iu b)(c - \iu d)}{c^2 + d^2} = 
                    \frac{ac + bd}{c^2 + d^2} + \iu \frac{bc - ad}{c^2 + d^2}
            \]
        \item For 
            \[
                z = \sqrt{\frac{1}{2}(a + \sqrt{a^2 + b^2})} + \iu \sqrt{\frac{1}{2}(-a + \sqrt{a^2 + b^2})} 
            \]
            it follows
            \[
                z^2 = \frac{1}{2}(a + \sqrt{a^2 + b^2}) - \frac{1}{2}(-a + \sqrt{a^2 + b^2}) +
                    \iu 2\sqrt{\frac{1}{4} (\sqrt{a^2 + b^2}^2) - a^2} = a + \iu b
            \]
    \end{enumerate}
\end{lemma}
\bigskip

\begin{lemma}[Binomial Theorem]\hfill
    \begin{enumerate}
        \item For the binomial coefficient Pascal's identity holds
			\[
				\binom{n}{k - 1} + \binom{n}{k} = \binom{n + 1 }{k }   
			\]  
        \item The following equation states the binomial identity
			\[
				{(a + b)}^n = \sum_{k=0}^{n} \binom{n}{k} a^{k} b^{n - k} = \sum_{k=0}^{n} \binom{n}{k} a^{n -k} b^{k}
			\]  
        \item For \( a = 1 \) follows
			\[
				{(1 + x)}^n = \sum_{k=0}^{n} \binom{n}{k} x^{k}
			\]  
    \end{enumerate}
\end{lemma}

\begin{proof}
It is
\[
	\binom{n}{k} + \binom{n}{k - 1} = \frac{n!}{k!(n - k)!} + \frac{n!}{(k - 1)!(n - k + 1)!}
		= \frac{n!(n + 1 - k) + n!k!}{k!(n + 1- k)!} = \binom{n + 1}{k}
\]
Furthermore by using induction
\[
    \begin{split}
		{(a + b)}^{n + 1}	& = \sum_{k=0}^{n} \binom{n}{k} a^{k + 1} b^{n - k} + 
								\sum_{k=0}^{n} \binom{n}{k} a^{k} b^{n + 1 - k} \\
							& = \sum_{k=1}^{n + 1} \binom{n}{k - 1} a^{k} b^{n + 1 - k} + 
								\sum_{k=0}^{n} \binom{n}{k} a^{k} b^{n + 1 - k} \\
							& = \sum_{k=0}^{n + 1} \binom{n + 1}{k} a^{k} b^{n + 1 - k}
    \end{split}
\]  
\end{proof}
\bigskip
