
\newpage
\section{One Dimensional Minimization and Direct Search}


\begin{definition}[Unimodal Function]\label{def:unimodal_fnc}
A function \( f : [a,b] \to \R \) is called unimodal if there exists a \( \xi \in [a,b] \), so that
\( f \) is strictly decreasing in \( [a, \xi] \) and strictly increasing in \( [\xi, b] \).
\end{definition}
\bigskip

In fact \( \xi \) is the unique minimum of \( f \) in \( [a, b] \). According to the definition, 
for \( a \le x < y \le b \) we have 
\[
    f(x) > f(y) \text{ for } x, y \in [a, \xi) \text{ and } f(x) < f(y) \text{ for }  x, y \in (\xi, b]
\]
Thus
\[
    \xi \in [a, y] \text{ if } f(x) < f(y) \text{ and } \xi \in [x, b] \text{ if } f(x) \ge f(y)
\]
Let now \( [a_0, b_0] = [a, b] \) and \(a_0 \le x_0 < y_0 \le b_0 \). Furthermore
\[  
    [a_{k + 1}, b_{k + 1}] = 
        \begin{cases}
            [a_k, y_k] & \text{ if } f(x_k) < f(y_k)  \\
            [x_k, b_k] & \text{ if } f(x_k) \ge f(y_k)
        \end{cases}
\]
for \( a_k \le x_k < y_k \le b_k \) respectively. It follows that \( [a_k, b_k] \supset [a_{k + 1}, b_{k + 1}] \) 
is a decreasing series of intervals.

Consider a partioning of the interval \( [0, 1] \) where the following relations hold 
\[
     x = 1 - y \text{ and } \\ \frac{1}{y} = \frac{y}{x}
\]

Then \( 1 - y = y^2 \) and solving the quadratic equation \( y^2 + y = 1 \) gives
\[ 
     y = \frac{\sqrt{5} - 1}{2} \approx 0.6180 < 1
\]
Furthermore
\[  
    (b_{k + 1} - a_{k + 1}) =  y(b_k - a_k)
\]
and so the interval converges to \( \xi \).
\bigskip


\begin{algorithm}[Golden Section Search]\label{algo:golden_section_search}
Primitive implementation onf the golden section search in Python.
\hfill\bigskip
\end{algorithm}
\inputminted[fontsize=\small, framesep=0.35cm, frame=lines, python3=true]{python}{golden_section.py}
\bigskip

\begin{tikzpicture}
\begin{axis}[axis lines=left, xlabel=$x$, ylabel={$f(x)$}, xmin=-5, xmax=5, ymin=0, ymax=5]
\addplot[domain=-10:10,samples=100,color=red]{(x - 2)^2 + 1};
\addlegendentry{$ {(x - 2)}^2 + 1 $}
\addplot[domain=-10:10, samples=100, color=blue]{x^2};
\addlegendentry{$ x^2 $}
\end{axis}
\end{tikzpicture}


% \begin{tikzpicture}
% \begin{axis}[
%     title=Exmple using the mesh parameter,
%     hide axis,
%     colormap/cool,
% ]
% \addplot3[
%     mesh,
%     samples=50,
%     domain=-8:8,
% ]
% {sin(deg(sqrt(x^2+y^2)))/sqrt(x^2+y^2)};
% \addlegendentry{$\frac{sin(r)}{r}$}
% \end{axis}
% \end{tikzpicture}