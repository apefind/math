
\newpage
\section{Introduction}

\theorem[Fermat Stationary Points]\label{thm:fermat_stationaryl_point}
Let \( \Omega \subseteq \R \) be open and \( f \in C^1(\Omega) \). If \( x^* \in \Omega \) is an stationary point then
\( f^\prime(x^*) = 0 \).

\proof{}
Assume \( x^* \) is the minimum of \( f \) in \( \Omega \) and let \( f^\prime(x^*) > 0 \). 
Since \( f \in C^1(\Omega) \) there exist \( \eps, \delta > 0 \), so that for \( |h| \le \eps \)
\[
    \frac{f(x^* + h) - f(x^*)}{h} > \delta
\]
Pick a negative \( h \in [-\eps, 0) \). Then 
\[
     f(x^* + h) < f(x^*) +  \delta h < f(x^*) 
\]
and \( x^* \) cannot be the minimum. Analog for maximum with a positive \( h \), then apply to \( -f \).
\bigskip


\lemma[Direction Derivative]\label{lemma:direction_derivative}
Let \( \Omega \subseteq \Rn \) be open, \( f \in C^1(\Omega) \) and \( d \in \Rn \) be a feasable direction 
at \( x \in \Omega \). Then
\[
    \frac{\partial f}{\partial d}(x) = \gradient{f(x)}^T d
\] 
\bigskip


\theorem[First Order Necessary Condition]\label{thm:fonc}
Let \( \Omega \subseteq \Rn \) be open and \( f \in C^1(\Omega) \). If \( x^* \in \Omega \) is a local minimzer then
\( \gradient f(x^*) = 0 \).

\proof{}
Let \( h \in \Rn \) and \( \delta > 0 \) so that \( x^* + th \in \Omega \) for all \( t \in (-\delta, \delta) \). 
Then \( 0 \) is local minimizer for \( \varphi(t) = f(x^* + th) \) and
\[
    \varphi^\prime(0) = {\gradient f(x^*)}^T h = 0
\]
Now let \( h = \gradient f(x^*) \). 
\bigskip
