
\newpage
\section{Calculus}

\subsection{Differentation and Integration}
\bigskip


\begin{lemma}[Simple Calculations]\hfill
    \begin{enumerate}
        \item For \( 1 = x x^{-1} \) the product rule yields \( 0 = x^{-1} + x(x^{-1})' \). Hence
              \[
                  \frac{d}{dx} x^{-1} = -\frac{1}{x^2}
              \]
        \item Similarly \( x = \sqrt{x}^2 \) and \( 1 = 2 \sqrt{x} \sqrt{x}' \) and so
              \[
                  \frac{d}{dx} \sqrt{x} = \frac{1}{2\sqrt{x}}
              \]
        \item It is ok
              \[
                  \frac{d}{dx} x^n = nx^{n - 1}
              \]
              since via induction the product rule yields
              \[
                  \frac{d}{dx} x^n = \frac{d}{dx} xx^{n -1} = x^{n -1} + \frac{d}{dx} x^{n - 1} =
                  x^{n -1} + (n - 1)x^{n - 1}  = nx^{n - 1}
              \]
        \item Again, applying the product rule gives
              \[
                  \left(\frac{1}{g}\right)' = \left(\frac{1}{x} \circ g\right)' = -\frac{g'}{g^2}
              \]
              and the quotient rule
              \[
                  \left(\frac{f}{g}\right)' = \frac{f'}{g} + f \left(\frac{1}{g} \right)' =
                  \frac{f'}{g} -\frac{fg'}{g^2} = \frac{gf'- fg'}{g^2}
              \]
        \item Also \( x = f \circ f^{-1} \) and \( 1 = (f^{-1})'f' \circ f^{-1} \). Thus
              \[
                  (f^{-1})' = \frac{1}{f' \circ f^{-1}}
              \]
              where defined. Especially for \( x \ne 0 \)
              \[
                  \log'(x) = \frac{1}{\exp'(\log(x))} = \frac{1}{x}
              \]

        \item \( (1 - q) (1 + q + q^2 + \cdots + q^n) = 1 - q + q - q^2 + q^2 - q^3 + \cdots + q^{n+1} \) gives
              \[
                  \sum_{k=0}^n q^k = \frac{1 - q^{n+1}}{1 - q} \text{ and }
                  \sum_{k=m}^n q^k = \frac{q^m - q^{n+1}}{1 - q}
              \]
    \end{enumerate}
\end{lemma}
\bigskip

\begin{remarks}Let \( f \) be differentiable at \( x_0 \). From the definition follows
    \begin{enumerate}
        \item \( f \) is \emph{Lipschitz continous} at \( x_0 \), e.g. there exists a \( L > 0 \), so that
              \( |f(x) - f(x_0)| < L|x - x_0| \) in some small enough environment of \( x_0 \).
        \item If there exists a sequence  \( x_n \to x_0 \) with \( f(x_n) = f(x_0) \) then \( f'(x) = 0 \).
              Consequently, if \( f'(x) \ne 0 \) then \( f(x) \ne f(x_0) \) in some local environment of \( x_0 \).
    \end{enumerate}
\end{remarks}
\bigskip

\begin{lemma}[Chain Rule]
    The chain rule for differentiable functions \( f \) and \( g \) yields
    \[
        (f \circ g)'(x) = f'(g(x))\, g'(x)
    \]

    \begin{proof}
        For \( g'(x_0) \ne 0 \) there exists a local environment of \( x_0 \) where
        \[
            \frac{f(g(x)) - f(g(x_0))}{x - x_0} =
            \frac{f(g(x)) - f(g(x_0))}{g(x) - g(x_0)} \frac{g(x) - g(x_0)}{x - x_0}
        \]
        Otherwise the remark above gives
        \[
            \left| \frac{f(g(x)) - f(g(x_0))}{x - x_0} \right| \le
            L\left| \frac{g(x) - g(x_0)}{x - x_0}  \right|
        \]
        and \( (f \circ g)'(x_0) = 0 \).
    \end{proof}
\end{lemma}
\bigskip


\begin{lemma}[Exponential Function]\hfill
    \begin{enumerate}
        \item It is
              \[
                  \exp(x + y) = \exp(x)\exp(y)
              \]
              Hence
              \[
                  \exp(0) = 1
              \]
              \[
                  \exp(-x) = {\exp(x)}^{-1}
              \]
              \[
                  \exp(nx) = {\exp(x)}^n
              \]
        \item For the derivative
              \[
                  \exp'(x) = \sum_{k=0}^\infty \frac{1}{k!} (x^k)' = \sum_{k=0}^\infty \frac{1}{k!} kx^{k-1}
                  = \sum_{k=1}^\infty \frac{1}{(k-1)!} x^{k-1} = \exp(x)
              \]
    \end{enumerate}
\end{lemma}
\bigskip


\begin{lemma}[Sinus and Cosinus]\hfill
    \begin{enumerate}
        \item Sinus and Cosinus power series
              \[
                  \begin{split}
                      \cos(x) &= \sum_{k=0}^\infty \frac{{(-1)}^k}{2k!} x^{2k} \\
                      \sin(x) &= \sum_{k=0}^\infty \frac{{(-1)}^k}{(2k + 1)!} x^{2k + 1}
                  \end{split}
              \]
        \item Symmetry
              \[
                  \begin{split}
                      \cos(-x) &= \sum_{k=0}^\infty \frac{{(-1)}^k}{2k!} {(-x)}^{2k} = \cos(x) \\
                      \sin(x) &= \sum_{k=0}^\infty \frac{{(-1)}^k}{(2k + 1)!} {(-x)}^{2k + 1} = -\sin(x)
                  \end{split}
              \]
        \item Derivatives
              \[
                  \begin{split}
                      \cos'(x) &= \sum_{k=1}^\infty \frac{{(-1)}^k}{(2k - 1)!} x^{2k - 1}
                      = \sum_{k=0}^\infty \frac{{(-1)}^{k + 1}}{(2k + 1)!} x^{2k + 1} = -\sin(x) \\
                      \sin'(x) &= \sum_{k=0}^\infty \frac{{(-1)}^k}{2k!} {x}^{2k} = \cos(x)
                  \end{split}
              \]
    \end{enumerate}
\end{lemma}
\bigskip


\begin{theorem}[Fermat Stationary Point]\label{thm:fermat_stationary_point}
    Let \( \Omega \subseteq \R \) be open and \( f \in C^1(\Omega) \). If \( x^* \in \Omega \) is local extremum
    then \( f'(x^*) = 0 \).
\end{theorem}

\begin{proof}
    Assume \( x^* \) is the minimum of \( f \) in \( \Omega \) and let \( f(x^*) > 0 \).
    Since \( f \in C^1(\Omega) \) there exist \( \eps, \delta > 0 \), so that for \( |h| \le \eps \)
    \[
        \frac{f(x^* + h) - f(x^*)}{h} > \delta
    \]
    Pick a negative \( h \in [-\eps, 0) \). Then   % chktex 9 
    \[
        f(x^* + h) < f(x^*) +  \delta h < f(x^*)
    \]
    and \( x^* \) cannot be the minimum. Analog for maximum with a positive \( h \), then apply to \( -f \).
\end{proof}
\bigskip


\begin{theorem}[Rolle]\label{thm:rolle}
    Let \( f \in C[a,b] \) with \( f(a) = f(b) \). If \( f \) is differentiable in \( (a, b) \) then
    there exists a \( \xi \in (a,b) \) with \( f'(\xi) = 0 \).
\end{theorem}

\begin{proof}
    Assume \( f \) is not constant. Since \( [a,b] \) is compact there exists either a global minimum or maximum
    \( \xi \in (a,b) \) and Theorem~\ref{thm:fermat_stationary_point} can be applied.
\end{proof}
\bigskip


\begin{theorem}[Mean Value]\label{thm:mean_value}
    Let \( f \in C[a,b] \) be differentiable in \( (a, b) \). Then there exists a \( \xi \in (a,b) \) with
    \[
        f'(\xi) = \frac{f(b) - f(a)}{b - a}
    \]
\end{theorem}

\begin{proof}
    Apply Theorem~\ref{thm:rolle} to
    \[
        g(x) = f(x) - \frac{f(b) - f(a)}{b - a} (x -a)
    \]
\end{proof}
\bigskip


\begin{remarks}\hfill
    \begin{enumerate}
        \item More generally choose any \( \varphi \in C^1[a,b] \) with \( \varphi(a) = 0 \) and
              \( \varphi(b) = f(b) - f(a) \). Set \( g(x) = f(x) - \varphi(x) \) to see there is a \( \xi \in (a,b) \)
              with \( f'(\xi) = \varphi'(\xi)\).
        \item Let \( f \) be differentiable in \( (a, b) \) with \( f' = 0 \). For any \( x, y \in (a, b) \)
              \[
                  0 = f'(\xi) = \frac{f(y) - f(x)}{y - x}
              \]
              and \( f \) is a constant.
        \item Another useful generalization: let \( \Omega \subseteq \Rn \) be open and \( f \in C^1(\Omega) \). For
              \( x, y \in \Omega \) define \( \varphi(t) = f(tx + (1 - t)y) \) and apply the chain rule for differentiation
              \[
                  \varphi'(\xi) = \gradient {f(\xi x + (1 - \xi)y)}^T(x - y) = f(x) - f(y)
              \]
        \item The Cauchy Schwarz inequality then yields
              \[
                  \|f(x) - f(y)\| \le \|\gradient f(\xi x + (1 - \xi)y)\| \|(x - y)\|
              \]
    \end{enumerate}
\end{remarks}
\bigskip


\begin{theorem}[Differentiation Theorem]\label{thm:differentiation}
    Let \( f \in C[a,b] \) and define
    \[
        F(x) = \int_a^x f(t)\,dt
    \]
    Then \( F \in C^1[a,b] \) with \( F'(x) = f(x) \) for \( x \in [a,b] \).
\end{theorem}

\begin{proof}
    Applying the Mean Value Theorem of Integration gives
    \[
        F(x + h) - F(x) =  \int_x^{x + h} f(t)\,dt = f(\xi) h
    \]
    for some \( \xi \in (x, x + h) \).
\end{proof}
\bigskip

\begin{theorem}[Fundamental Theorem of Calculus]\label{thm:fund_calculus}
    Let \( F \in C^1[a,b] \) with \( F' = f \)  Then
    \[
        F(b) -F(a) = \int_a^b f(t)\,dt
    \]
\end{theorem}
\bigskip


\begin{lemma}[Integration by Substitution]
    Let \( I \subseteq \R \) be an interval and \( f \in C(I) \). For \( \varphi \in C([a,b], I) \) it follows
    \[
        \int_{\varphi(a)}^{\varphi(b)} f(x)\,dx = \int_{a}^{b} f(\varphi(t))\varphi'(t)\,dt
    \]
\end{lemma}
\begin{proof}
    Let \( F \in C^1(I) \) with \( F' = f \). Then the chain rule for differentiation yields
    \[
        \begin{split}
            \int_{\varphi(a)}^{\varphi(b)} f(x)\,dx
            &= F(\varphi(b)) - F(\varphi(a)) \\
            &= F\circ\varphi(b) - F\circ\varphi(a) \\
            &= \int_{a}^{b} (F\circ\varphi)'(t)\,dt \\
            &= \int_{a}^{b} f(\varphi(t))\varphi'(t)\,dt
        \end{split}
    \]
\end{proof}
\bigskip

\begin{examples}\hfill
    \begin{enumerate}
        \item For \( \varphi(x) = x^2 + 1 \) it is \( \varphi(0) = 1 \) and \( \varphi(2) = 5 \). Thus
              \[
                  \int_0^2 x\cos(x^2 + 1)\,dx
                  = \frac{1}{2} \int_0^2 2x\cos(x^2 + 1)\,dx
                  = \frac{1}{2} \int_1^5 \cos(t)\,dt
                  = \frac{1}{2} (\sin(5) - \sin(1))
              \]
        \item Consider \( \varphi(x) = \sin(x) \) where \( \varphi(0) = 0 \) and \( \varphi(\pi / 2) = 1 \).
              Since \( \cos(t) = \sqrt{1 - \sin^2(t)} \) it follows
              \[
                  \int_0^1 \sqrt{1 - x^2}\,dx
                  = \int_{\cos(0)}^{\cos(\pi/2)} \sqrt{1 - x^2}\,dx
                  = \int_0^{\pi/2} \sqrt{1 - \sin^2(t)}\cos(t)\,dt
                  = \int_0^{\pi/2} \cos^2(t)\,dt
              \]
        \item Let \(f \in C[a,b] \) and \( \varphi(x) = a + t(b - a) \). Then
              \[
                  \int_a^b f(x)\,dx = (b - a)\int_0^1 f(a + t(b - a))\,dt
              \]
        \item Let \(f(x) = x^n \) and \( \varphi(x) = t^m \). As expected
              \[
                  \int_0^1 x^n\,dx
                  = \int_0^1 t^{nm} m t^{m - 1}\,dt
                  = m\int_0^1 t^{m(n + 1) - 1}\,dt
                      = {\left[\frac{m}{m(n + 1)} t^{m(n + 1)}\right]}_0^1
                  = \frac{1}{n + 1}
              \]
    \end{enumerate}
\end{examples}
\bigskip


\subsection{Multivariable Derivative}
\bigskip

\begin{definition}
    Let \( \Omega \subseteq \Rn \) be open and \( x \in \Omega \).
    \begin{enumerate}
        \item \( f : \Omega \to \Rm \) is called \emph{differentiable} at \( x \) if there exists
              a linear mapping \( A: \Rn \to \Rm \), so that
              \[
                  \lim_{h \to 0} \frac{\|f(x + h) - f(x) - Ah\|}{\|h\|} = 0
              \]
              Here \( Df(x) = A \)
              is called the \emph{derivative} of \( f \) at \( x \).
        \item In this context linear mappings are used synonymously with their corresponding \emph{Jacobian matrices}.
        \item For \( f : \Omega \to \R \) the derivative is called the \emph{gradient} of \( f \) at \( x \)
              \[
                  \gradient f(x) = Df(x)
              \]
    \end{enumerate}
\end{definition}
\bigskip


\begin{remarks}\hfill
    \begin{enumerate}
        \item For a linear mapping \( A \) and an arbitrary \( v = th \in \Rn \) it is
              \[
                  \frac{\|Av\|}{\|v\|} = \frac{\|Ah\|}{\|h\|}
              \]
              Hence
              \[
                  \frac{\|Av - Bv\|}{\|v\|} = \frac{\|Ah - Bh\|}{\|h\|}\le
                  \frac{\|f(x + h) - f(x) - Ah\|}{\|h\|} + \frac{\|f(x + h) - f(x) - Bh\|}{\|h\|}
              \]
              and the derivative is well defined.
        \item As a consequence, if already \( f(x + h) - f(x) - Ah = 0 \) then \( Df(x) = A \).
              Hence \( Df(x) = 0 \) for a constant  \( f \) and \( f(x) = Ax \) gives \( Df(x) = A \).
        \item Since all norms on \( \Rn \) are equivalent the differentiability of \( f = (f_1, f_2, \dots, f_n) \)
              is eqivalent to the differentiability of all its components.
    \end{enumerate}
\end{remarks}
\bigskip


\begin{definition}
    Let \( \Omega \subseteq \Rn \) be open and \( x \in \Omega \).
    \begin{enumerate}
        \item For \( f : \Omega \to \R \) and a normed vector \( v \in \Rn \) the limit
              \[
                  \frac{\partial f}{\partial v}(x) = \lim_{t \to 0} \frac{f(x + tv) - f(x)}{t}
              \]
              is called the \emph{directional derivative} of \( f \) at \( x \)
              with respect to the direction \( v \). 
        \item The \emph{partial derivates} are the directional derivatives with respect to the
              unit vectors \( e_j \)
              \[
                  \frac{\partial f}{\partial x_j}(x) = \frac{\partial f}{\partial e_j}(x)
              \]
    \end{enumerate}
\end{definition}
\bigskip


\begin{examples}\hfill
    \begin{enumerate}
        \item Let
              \[
                  f(x,y) = \frac{xy(x^2 - y^2)}{x^2 + y^2} = \frac{y(x^3 - xy^2)}{x^2 + y^2}
              \]
              Then
              \[
                  \frac{\partial f}{\partial x}(x,y)
                  = \frac{(3yx^2 - y^3)(x^2 - y^2) - 2xy(x^3 - xy^2)}{{(x^2 + y^2)}^2}
                  = \frac{yx^4 + 4y^3x^2 - y^5}{{(x^2 + y^2)}^2}
              \]
              Since \( f(y,x) = - f(x,y)\) it follows
              \[
                  \frac{\partial f}{\partial y}(x,y) = \frac{x^5 - 4x^3y^2 - xy^4}{{(x^2 + y^2)}^2}
              \]

    \end{enumerate}
\end{examples}
\bigskip


\begin{examples}\hfill
    \begin{enumerate}
        \item Consider the norm \( r(x) = \|x\| = \sqrt{x_1^2 + x_2^2 \dots x_n^2} \). Then the chain rule yields
              \[
                  \frac{\partial r}{\partial x_j}(x) = \frac{2 x_j}{2 \sqrt{x_1^2 + x_2^2 \dots x_n^2}}
                  = \frac{x_j}{\|x\|}
              \]
              and \( r \) is differentiable on \( \Rn {\setminus} \{ 0 \} \).
        \item For \( f \) differentiable follows
              \[
                  \frac{\partial f}{\partial x_j}(r(x)) = x_j \frac{f'(r(x))}{\|x\|}
              \]
    \end{enumerate}
\end{examples}
\bigskip


\begin{lemma}[Directional Derivative]\label{lemma:directional_derivative}
    Let \( \Omega \subseteq \Rn \) be open and \( f \in C^1(\Omega) \). Then
    \[
        \frac{\partial f}{\partial d}(x) = \gradient{f(x)}^T d
    \]
    for any \( d \in \Rn \).
\end{lemma}

\begin{proof}
    Let \( \varphi(t) = f(x + td) \). Then \( \varphi \in C^1[-\eps, \eps ] \) for some \( \eps > 0 \)
    and the chain rule yields
    \[
        \varphi'(t) = {\gradient f(x + td)}^T d
    \]
    Hence
    \[
        \varphi'(0) = \lim_{t \to 0} \frac{\varphi(x + td) - \varphi(0)}{t} =
        \lim_{t \to 0} \frac{f(x + td) - f(x)}{t} = {\gradient f(x)}^T d
    \]
\end{proof}
\bigskip


\begin{remarks}\hfill
    \begin{enumerate}
        \item A similar proposition holds under the weaker assumption that \( d \) is a only feasable direction
              for \( f \) in \( x \)
        \item For \( d = \gradient{f(x)}/\| \gradient{f(x)}\| \) it follows that
              \[
                  \frac{\partial f}{\partial d}(x) = \|\gradient{f(x)}\| > 0
              \]
              and for any other \( d \in \Rn \) with \( \|d\| = 1 \) the Cauchy Schwarz inequality yields
              \[
                  \left|\frac{\partial f}{\partial d}(x)\right| = |\gradient{f(x)}^T d|
                  \le \|\gradient{f(x)}\| \|d\| =\|\gradient{f(x)}\|
              \]
              Hence \( \gradient{f(x)} \) is the direction of the greatest ascent and respectively,
              \( -\gradient{f(x)} \) is the direction of the greatest descent.
    \end{enumerate}
\end{remarks}
\bigskip


\begin{theorem}[First Order Necessary Condition]\label{thm:fonc}
    Let \( \Omega \subseteq \Rn \) be open and \( f \in C^1(\Omega) \). If \( x^* \in \Omega \) is a local minimzer then
    \( \gradient f(x^*) = 0 \).
\end{theorem}

\begin{proof}
    Let \( h \in \Rn \) and \( \delta > 0 \) so that \( x^* + th \in \Omega \) for all \( t \in (-\delta, \delta) \).
    Then \( 0 \) is local minimizer for \( \varphi(t) = f(x^* + th) \) and
    \[
        \varphi'(0) = {\gradient f(x^*)}^T h = 0
    \]
    Now let \( h = \gradient f(x^*) \).
\end{proof}
\bigskip


\begin{theorem}[Banach Fixed-Point Theorem]\label{thm:banach_fix_point}
    Let \( X \) be a Banach space and \( f \in C(X,X) \) a contraction
    \[
        \|f(x) - f(y)\| \le q \|x - y\| \text{ for all } x, y \in X
    \]
    for some \( 0 < q < 1 \). Then there exists a unique fix point \( x^* \in X \) with
    \[
        f(x^*) = x^*
    \]
    Furthermore for any \( x_0 \in X \) the sequence defined by
    \[
        x_{n+1} = f(x_n)
    \]
    converges aganist \( x^* \).
\end{theorem}

\begin{proof}
    Since \( \| x_{n+1} - x_n\| =\| f(x_n) - f(x_{n-1})\| \le q\| x_n - x_{n-1}\| \) it follows, that
    \[
        \| x_{n+1} - x_n\| \le q^n  \|x_1 - x_0\|
    \]
    Furthermore
    \[
        \| x_n - x_m\| \le \sum_{k=m}^n q^k \|x_1 - x_0\| = \frac{q^m - q^{n+1}}{1 - q} \|x_1 - x_0\|
    \]
    and \( (x_n) \) is a Cauchy sequence. For its limit \( x^* \) it follows
    \[
        x^* = \lim_{n\to\infty} x_{n+1} = \lim_{n\to\infty} f(x_n) = f(x^*)
    \]
    For any other \( y^* \in X \) with \( f(y^*) = y^* \) it follows, that
    \[
        \|x^* - y^*\| = \|f(x^*) - f(y^*)\| \le q \|x^* - y^*\|
    \]
    and therefore \( x^* = y^*\).

\end{proof}
\bigskip

