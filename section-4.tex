
\newpage
\section{Neural Networks}

\subsection{The Perceptron}

\begin{definition}[Activation Functions]\hfill
    \begin{enumerate}
		\item The \emph{Heaviside} function \( H: \R \to \{ 0, 1 \} \) is defined as
			\[
				H(x) = \left\{
					\begin{array}{ll}
						1 & \text{ for } x > 0 \\
						0 & \text{ for }x \le 0
					\end{array} 
				\right.
			\]
		\item The \emph{sigmoid} function \( \sigmoid \in C^\infty(\R) \) is defined as 
			\[
				\sigmoid(x) = \frac{1}{1 + e^{-x}}
			\]
    \end{enumerate}
\end{definition}
\bigskip


\begin{figure}[H]
	\centering
	\plotsigmoid
	\caption{The sigmoid function $ \sigmoid(x) = \frac{1}{1 + e^{-x}} $}\label{fig:sigmoid}
\end{figure}
\bigskip


\begin{remarks}\hfill
    \begin{enumerate}
		\item Since the heaviside function is not continous and therefore not differentiable at \( 0 \)
			the sigmoid function is often considered its smooth counterpart
		\item The definition of the sigmoid function yields \( 0 < \sigmoid(x) <  1 \) as well as 
			\( \sigmoid(x) \to 0 \) for \( x \to -\infty \) and \( \sigmoid(x) \to 1 \) for \( x \to \infty \)
		\item The quotient rule yields
			\[
				\sigmoid'(x) 
					= -\frac{-e^{-x}}{(1 + e^{-x})^2}
					= \sigmoid(x) \frac{1 + e^{-x} - 1}{1 + e^{-x}}
					= \sigmoid(x)(1 - \sigmoid(x))
			\]
			and \( \sigmoid \) is monotonically increasing over its domain
    \end{enumerate}
\end{remarks}
\bigskip


\begin{definition}[Binary Classifiers]
	Let \( X \subset \R^n \) be the union of two disjoint finite sets \( X = M \uplus N \).
    \begin{enumerate}
		\item A \emph{binary classification problem} is the task to find a mapping \( f: X \to \{ 0, 1 \} \) with
			\[
				f(x) = 1 \text{ for } x \in M \text{ and } f(x) = 0 \text{ for } x \in N
			\]
			\( f \) then is called a \emph{binary classifier} for \( X \)
		\item \( X \) is called \emph{separable} if there exists a \emph{weight vector} \( w \in \R^n \) 
			and a \emph{bias} \( b \in \R \) so that
				\[
						wx + b > 0 \text{ for } x \in M  \text{ and } 
						wx + b < 0 \text{ for } x \in N
				\]
		\item The weight \( w \) and the bias \( b \) are called \emph{solution to the classification problem}. 
		    They implicitely define a binary classifier via
				\[
					f(x) = H(wx + b)
				\]
		\item The set \( \{ x \in \R^n | wx + b = 0 \} \) is called a \emph{hyperplane}
    \end{enumerate}
\end{definition}
\bigskip


\begin{examples}\hfill
    \begin{enumerate}
        \item Let \( X = \{ 0, 1 \} \times \{ 0, 1 \} \) and consider the \emph{and} operator 
		    \( f(1, 1) = 1 \) and \( f(x, y) = 0 \) elsewhere. Then \( w = (3, 3) \) and \( b = -5 \) 
			yield a solution
        \item Again let \( X = \{ 0, 1 \} \times \{ 0, 1 \} \) and \( f(1, 0) = f(0, 1) = 1 \) and 
			\( f(0, 0) = f(1, 1) = 0 \), the \emph{xor} operator. Thus for any weight \( (w_1, w_2) \) and 
			any bias \( b \)
				\[
					\begin{split}
						w_1 + b > 0 & \text{ and } w_2 + b > 0 \\
						w_1 + w_2 + b \le 0 & \text{ and } b \le 0
					\end{split}
				\]
			Adding two equations respectively shows that there is no solution
		\item The bias can be integrated into the weight vector via \( w' = (w, b) \in \R^{n + 1} \) and 
			\( x' = (x, 1) \in \R^{n + 1} \). Separability then reduces to
				\[
					w'x' > 0
				\]
	\end{enumerate}
\end{examples}
\bigskip


\subsubsection*{Geometrical Interpretation}
The idea for the perceptron most likely has its origin in a simple geometrical observation.
Recall that for \( x, y \in \R^n \) the dot product can be expressed as 
\[
	xy = \|x\| \|y\| \cos \alpha 
\]
where \( \alpha \) is the angle between the two vectors. Hence the product is positive 
if the angle is less than \( \ang{90} \) degrees and negative if the angle is between \( \ang{90} \) 
and \( \ang{180} \) degrees
\[
	xy > 0 \text{ for } 0 \le \alpha < \pi / 2 \text{ and } xy < 0 \text{ for } \pi / 2 < \alpha \le \pi
\]
Note, that the sign does not dependend on lengths of the vectors, but solely on the angle.

For any two vectors it is easy enough to find a weight that satisfies \( wx > 0 \) and \( wy > 0 \). 
Generally \( w = x + y \) is a good guess, but not always correct as shown below in 
\hyperref[fig:vectorangle]{Figure \ref*{fig:vectorangle}}. 

\bigskip
\begin{figure}[H]
	\centering
	\plotvectorangle
	\caption{Dot Product and Angle}\label{fig:vectorangle}
\end{figure}
\bigskip

But, the more similar the lengths of the two vectors are the more likely \( x + y \) works. 
An iterative approach is to increase \( w = x + y \) in the direction of vector with the angle greater 
than \( \ang{90} \) degrees.

\begin{examples}\hfill
    \begin{enumerate}
        \item Let \( x = (-1, 1) \) and \( y = (6, 1) \). Then 
			\[ 
				\begin{align*}
					w_0 &= (5, 2) & w_0x &= -3 < 0 & w_0y &= 32 > 0 \\
					w_1 &= (4, 3) & w_1x &= -1 < 0 & w_1y &= 27 > 0 \\
					w_2 &= (3, 4) & w_2x &= 1 > 0 & w_2y &= 22 > 0 \\
				\end{align*}
			\]
        \item Let \( x = (4, -6) \) and \( y = (-10, 5) \). Then 
			\[ 
				\begin{align*}
					w_0 &= (-6, -1) & w_0x &= -18 < 0 & w_0y &= 55 > 0 \\
					w_1 &= (-2, -7) & w_1x &= 34 > 0 & w_1y &= -15 < 0 \\
					w_2 &= (-12, -2) & w_2x &= -36 < 0 & w_2y &= 110 > 0 \\
					w_3 &= (-8, -8) & w_3x &= 16 > 0 & w_3y &= 40 > 0 \\
   				\end{align*}
			\]
    \end{enumerate}
\end{examples}
	

\begin{algorithm}[Weight]\label{algo:weight}
\end{algorithm}
\inputminted[fontsize=\small, framesep=0.35cm, frame=lines, python3=true]{python}{python/weight.py}
\bigskip


% \[
% 	wx = \left( \frac{x}{\|x\|} + \frac{y}{\|y\|)} \right)x = \|x\| + \|x\|\cos\alpha > 0
% \]
% The same is true for \( wy \).


% Let \( X = M \uplus N \) be separable. Then
% \[
% 	w = \sum_{x \in N} \frac{x}{\|x\|} - \sum_{y \in M} \frac{y}{\|y\|}
% \]
% is a weight that is a solution to the classification problem.
\bigskip


\begin{algorithm}[Perceptron]\label{algo:perceptron}
\end{algorithm}
\inputminted[fontsize=\small, framesep=0.35cm, frame=lines, python3=true]{python}{python/perceptron.py}
\bigskip


%\subsection{The Backtracking Algorithm}


