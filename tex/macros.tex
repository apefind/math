
\newcommand{\sometext}{\bigskip\rm}
\newcommand{\eps}{\varepsilon}
\newcommand{\gradient}{\nabla}
\newcommand{\hessian}{\gradient^2}
\newcommand{\argmin}{\operatorname*{arg\,min}}
\newcommand{\argmax}{\operatorname*{arg\,max}}
% \newcommand{\R}[1][]{\mathbb{R}^{\ifx&#1&#1\else\fi}}
\newcommand{\R}{\mathbb{R}}
\newcommand{\Rn}{\mathbb{R}^n}
\newcommand{\Rm}{\mathbb{R}^m}
\newcommand{\Rnn}{\mathbb{R}^{n \times{} n}}
\newcommand{\Rnm}{\mathbb{R}^{n \times{} m}}
\newcommand{\Rmm}{\mathbb{R}^{m \times{} m}}
\newcommand{\C}{\mathbb{C}}
\newcommand{\iu}{\mathrm{i}\mkern1mu}
\newcommand{\eu}{\mathrm{e}\mkern1mu}
\newcommand{\boundary}{\partial}
\newcommand{\pathlength}{\mathrm{L}}
\newcommand{\sigmoid}{\sigma}
% \renewcommand{\qedsymbol}{}

% \swapnumbers
\theoremstyle{plain}
\newtheorem{theorem}{Theorem}[section]
\newtheorem{corollary}[theorem]{Corollary}
\newtheorem{lemma}[theorem]{Lemma}
\newtheorem{formulas}[theorem]{Formulas}
\newtheorem{algorithm}[theorem]{Algorithm}

\theoremstyle{definition}
\newtheorem{definition}[theorem]{Definition}
\newtheorem{definitions}[theorem]{Definitions}

% \theoremstyle{remark}
\newtheorem*{remark}{Remark}
\newtheorem*{remarks}{Remarks}
\newtheorem*{example}{Example}
\newtheorem*{examples}{Examples}
\newtheorem*{exercise}{Exercise}




