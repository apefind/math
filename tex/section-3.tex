\newpage
\section{Functional Analysis}

\subsection{Hilbert Spaces}
\bigskip

\begin{definition}[Inner Product]
    Let \( E \) be a vector space with \( x, y, z \in E \) and \( \lambda \in \C \). A mapping
    \( \innerprod{}{}: E\times E \to \C \) is called \emph{inner product} if
    \begin{enumerate}
        \item \( \innerprod{x + y}{z} = \innerprod{x}{z} + \innerprod{y}{z} \)
        \item \( \innerprod{\lambda x}{y} = \lambda \innerprod{x}{y}\)
        \item \( \innerprod{x}{y} = \conjugate{\innerprod{y}{x}} \)
        \item \( \innerprod{x}{x} \ge 0 \) and \( \innerprod{x}{x} = 0 \) iff \( x = 0 \)
    \end{enumerate}
\end{definition}
\bigskip


\begin{lemma}
    \hfill
    \begin{enumerate}
        \item \( \|x\| = \innerprod{x}{x}^{1/2} \) defines a norm on \( E \)
              which makes the inner product is a continous mapping.
        \item \( |\innerprod{x}{y}| \le \|x\| \|y\| \)
        \item \( {\|x - y\|}^2 = {\|x\|}^2 - 2\Re\innerprod{x}{y} + {\|y\|}^2 \)
        \item \( {\|x + y\|}^2 + {\|x - y\|}^2 = 2{\|x\|}^2 + 2{\|y\|}^2 \)
    \end{enumerate}
\end{lemma}
\begin{proof}
    It is
    \[
        \begin{split}
            {\|x + y\|}^2 & = \innerprod{x + y}{x + y}  = \innerprod{x}{x + y} + \innerprod{y}{x + y}
            = \innerprod{x}{x} + \innerprod{x}{y} + \innerprod{y}{x} + \innerprod{y}{y} \\
            & = {\|x\|}^2 + \innerprod{x}{y} + \conjugate{\innerprod{y}{x}} + {\|y\|}^2
            = {\|x\|}^2 + 2\Re\innerprod{x}{y} + {\|y\|}^2
        \end{split}
    \]
    Hence
    \[
        {\|x + y\|}^2 \le {\|x\|}^2 + 2|\Re\innerprod{x}{y}| + {\|y\|}^2
        \le {\|x\|}^2 +  2\|x\| \|y\| + {\|y\|}^2 = {(\|x\| + \|y\|)}^2
    \]
    which proves the triangle inequality.
\end{proof}
\bigskip


\subsection{The Hahn-Banach Theorem}
\bigskip

\begin{remarks}
    \hfill
    \begin{enumerate}
        \item Let \( f = u + \iu v\) be a complex valued linear functional.
              Then \( f(\iu x) = \iu f(x) \) gives
              \[
                  u(\iu x) = -v(x) \text{ and } v(\iu x) = \iu u(x)
              \]
              Thus
              \[
                  f(x) = u(x) - \iu u(\iu x)
              \]
        \item Assume \( |u(x)| \le C\|x\| \) and choose \( t \in \R \) so that \( |f(x)| = e^{\iu t}f(x) \).
              Then
              \[
                  |f(x)| = e^{\iu t}f(x) = f(e^{\iu t}x) = u(e^{\iu t}x) \le C\|e^{\iu t}x\| = C\|x\|
              \]
        \item
              See also the \hyperref[thm:thm_cauchy_riemann_eqations]{Cauchy-Rieman
                  Eqautions~\ref*{thm:thm_cauchy_riemann_eqations}}.
    \end{enumerate}
\end{remarks}
\bigskip


\begin{theorem}[Hahn-Banach Theorem]\label{thm:hahn_banach}
    Let \( E \) be a normed space and \( F \) a linear subspace. For any linear functional
    \( f \in F^* \) there exists a \( \tilde{f} \in E^* \) so that
    \[
        \tilde{f}|_F = f \text{ and } \|\tilde{f}\| = \|f\|
    \]
\end{theorem}

\begin{proof}
    Assume that \( f: E \to \R \) is a real valued linear functional and \( x' \in E \setminus F \).
    Consider the subspace \( F' = \{x + \lambda x' \,|\, x \in F \} \) and define
    \[
        f'(x + \lambda x') = f(x) + \lambda r
    \]
    for some \( r \in \R \).  As sum of two linear functionals \( f' \) again is a linear functional with
    \( \|f'\| \ge \|f\| \).

    Determine \( \lambda \), so that the inequality holds.

    Using Zorn's Lemma allows the extension to \( E \) while the remarks above can be used to extend
    to complex valued linear functionals.
\end{proof}
\bigskip


\begin{theorem}[Separation of convex sets]\label{thm:separation}
    Let \( E \) be a real normed space and \( A, B \subset E \) disjoint convex subsets with \( A \)
    having nonempty interior. Then there exists a linear functional \( f \in E' \) so that \( f|_A \ge C \)
    and \( f|_B \le C \) for some constant. Furthermore \( f > C \) on the interrior of \( A \).

\end{theorem}

\begin{proof}
\end{proof}
\bigskip


\subsection{The Riesz Representation Theorem}
\bigskip

\begin{theorem}[The Riesz Representation Theorem]\label{thm:riesz_representation}
    Let \( X \) be a locally compact Hausdorff space and \( L \in C_c(X) \) a positive linear functional.
    Then ther exists a \( \sigma \)-algebra \( M \) in \( X \) and a unique positive measure
    \( \mu \) on \( M \) so that
    \[
        L(f) = \int_X f\,d\mu
    \]
    for every \( f \in C_c(X) \).
\end{theorem}
\bigskip
