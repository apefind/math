\newpage
\section{Neural Networks}

\subsection{The Perceptron}

\begin{definition}[Binary Classifiers]
    Let \( X \subset \R^n \) be the union of two finite disjoint sets \( X = M \cup N \).
    \begin{enumerate}
        \item A \emph{binary classification problem} is the task to find a mapping \( f: X \to \{ 0, 1 \} \) with
              \[
                  f(x) = \left \{
                  \begin{array}{ll}
                      1 & \text{ for } x \in M \\
                      0 & \text{ for } x \in N
                  \end{array}
                  \right.
              \]
              \( f \) then is called a \emph{binary classifier} for \( X \)
        \item \( X \) is called \emph{separable} if there exists a \emph{weight vector} \( w \in \R^n \)
              and a \emph{bias} \( b \in \R \) so that
              \[
                  \begin{split}
                      wx + b & > 0 \hspace{1em}\text{ for } x \in M \\
                      wx + b & < 0 \hspace{1em}\text{ for } x \in N
                  \end{split}
              \]
        \item The weight \( w \) and the bias \( b \) are called \emph{solution to the classification problem}.
              They implicitly define a binary classifier via
              \[
                  f(x) = \left \{
                  \begin{array}{ll}
                      1 & \text{ if } wx + b > 0 \\
                      0 & \text{ if } wx + b < 0
                  \end{array}
                  \right.
              \]
    \end{enumerate}
\end{definition}
\bigskip


\begin{examples}
    \hfill
    \begin{enumerate}
        \item Let \( X = \{ 0, 1 \} \times \{ 0, 1 \} \) and consider the \emph{and} operator
              \( f(1, 1) = 1 \) and \( f(x, y) = 0 \) elsewhere. Then \( w = (3, 3) \) and \( b = -5 \)
              yield a solution to the classification problem \( M = f^{-1}(1) \) and \( N = f^{-1}(0) \)
        \item Again let \( X = \{ 0, 1 \} \times \{ 0, 1 \} \) and \( f(1, 0) = f(0, 1) = 1 \) and
              \( f(0, 0) = f(1, 1) = 0 \), the \emph{xor} operator. Thus for any weight \( (w_1, w_2) \) and
              any bias \( b \)
              \[
                  \begin{split}
                      w_1 + b & > 0 \\
                      w_2 + b & > 0 \\
                      \\
                      w_1 + w_2 + b & \le 0 \\
                      b & \le 0
                  \end{split}
              \]
              Adding two equations respectively shows that there cannot be a solution
        \item The bias can be integrated into the weight vector via \( w' = (w, b) \in \R^{n + 1} \) and
              \( x' = (x, 1) \in \R^{n + 1} \). Separability then reduces to
              \[
                  w'x' > 0
              \]
    \end{enumerate}
\end{examples}
\bigskip


\subsubsection*{Geometrical Interpretation}
The idea for the perceptron most likely has its origin in a simple geometrical observation.
Recall that for \( x, y \in \R^n \) the dot product can be expressed as
\[
    xy = \|x\| \|y\| \cos\alpha
\]
where \( \alpha \) is the angle between the two vectors. Hence the product is positive
if the angle is less than \( \ang{90} \) degrees and negative if the angle is between \( \ang{90} \)
and \( \ang{180} \) degrees
\[
    \begin{split}
        xy & > 0 \hspace{1em}\text{ for } 0 \le \alpha < \pi / 2 \\
        xy & < 0 \hspace{1em}\text{ for } \pi / 2 < \alpha \le \pi
    \end{split}
\]
Note, that the sign does not depend on the vector lengths, but solely on the angle.

For any two vectors it is easy enough to find a weight that satisfies \( wx > 0 \) and \( wy > 0 \).
Generally \( w = x + y \) is a good guess, but not always correct as shown below in
\hyperref[fig:vectorangle]{Figure~\ref*{fig:vectorangle}}.

\bigskip
\begin{figure}[H]
    \centering
    \plotvectorangle{}
    \caption{Dot Product and Angle}\label{fig:vectorangle}
\end{figure}
\bigskip

But, the more similar the lengths of the two vectors are the more likely \( x + y \) works.
The actual threshold is given by the following
\bigskip

\begin{lemma}
    Let \( x, y \in \Rn \) with \( \|x\| < \|y\| \) and \( xy = \|x\| \|y\| \cos\alpha \). If
    \[
        \|x\| > -\|y\|\cos\alpha
    \]
    then \( x(x + y) > 0 \).
\end{lemma}

\begin{proof}
    Let \( x(x + y) = \|x\| \|x + y\| \cos\beta \). Since
    \[
        x(x + y) = xx + xy = {\|x\|}^2 + \|x\| \|y\| \cos\alpha
    \]
    it follows
    \[
        {\|x + y\|}\cos\beta = \|x\| + \|y\|\cos\alpha
    \]
    Hence \( \cos\beta > 0 \) if the inequality above holds.
\end{proof}
\bigskip


An iterative approach is to repeatedly increase \( w = x + y \) in the direction of the shorter vector 
aka the one with the angle greater than \( \ang{90} \) degrees.
\[
    w' = \left \{
    \begin{array}{ll}
        w + x & \text{ if } w x \le 0 \\
        w + y & \text{ if } w y \le 0 \\
    \end{array}
    \right.
\]

While this seems reasonable it is unclear whether the algorithm always yields a result after 
a finite number of iterations. The answer to this question will be given later as a special case 
of the \hyperref[thm:perceptron_convergence]{Perceptron Convergence Theorem~\ref*{thm:perceptron_convergence}}
\bigskip


\begin{algorithm}[Weight]\label{algo:weight}
\end{algorithm}
\inputminted[fontsize=\small, framesep=0.35cm, frame=lines, python3=true]{python}{python/weight.py}
\bigskip


\begin{examples}
    \hfill
    \begin{enumerate}
        \item Let \( x = (-1, 1) \) and \( y = (6, 1) \). Then
              \[
                  \begin{aligned}
                      w_0 & = (5, 2) & w_0x & = -3 & w_0y & = 32 \\
                      w_1 & = (4, 3) & w_1x & = -1 & w_1y & = 27 \\
                      w_2 & = (3, 4) & w_2x & = 1  & w_2y & = 22 \\
                  \end{aligned}
              \]
        \item Let \( x = (4, -6) \) and \( y = (-10, 5) \). Then
              \[
                  \begin{aligned}
                      w_0 & = (-6, -1)  & w_0x & = -18 & w_0y & = 55  \\
                      w_1 & = (-2, -7)  & w_1x & = 34  & w_1y & = -15 \\
                      w_2 & = (-12, -2) & w_2x & = -36 & w_2y & = 110 \\
                      w_3 & = (-8, -8)  & w_3x & = 16  & w_3y & = 40  \\
                  \end{aligned}
              \]
        \item
              Let \( w = x / \|x\| + y/ \|y\| \). Then
              \[
                  wx = \frac{{\|x\|}^2}{\|x\|} + \frac{yx}{\|y\|} = \|x\| + \|x\|\cos\alpha = (1 + \cos\alpha)\|x\| > 0
              \]
              and on the other hand \( wx = \|w\| \|x\| \cos\beta \). Similarily
              \( wy = (1 + \cos\alpha)\|y\| = \|w\| \|y\| \cos\gamma \). Hence
              \[
                  1 + \cos\alpha = \|w\|\cos\beta = \|w\|\cos\gamma
              \]
              and thus \( \beta = \gamma \). Also, as expected, \( wx > 0 \) and \( wy > 0 \)
        \item Task: for any given integer \( k \) find two vectors so that more than \( k \) steps are needed
    \end{enumerate}
\end{examples}
\bigskip


The following theorem is the generalization of the approach above for binary classifiers. 
Roughly sepaking the algorithm changes the weight only for misqualified vectors. The proof measures 
the change of angle against the change of length of the weight vector and provides an estimation 
for the maximumn number of required steps.
\bigskip

\begin{theorem}[Perceptron Convergence Theorem]\label{thm:perceptron_convergence}
    Let \( X = M \cup N \) be separable by \( w^*\in \Rn \). Define \( w_0 = 0 \)
    and repeat to iterate over all \( x \in X \) via
    \[
        w_{k + 1} = \left \{
        \begin{array}{ll}
            w_k + x & \text{ if } x \in M \text{ and } w_k x \le 0 \\
            w_k - x & \text{ if } x \in N \text{ and } w_k x \ge 0 \\
            w_k     & \text{ else }                                \\
        \end{array}
        \right.
    \]
    until no further changes occurr. Suppose \( \|x\| \le r \)  and \( w^*x \ge \delta > 0 \) 
    for \( x \in X \). Then the number of iterations before the algorithm stops is limited by
    \[
        k\le \frac{r^2}{\delta^2}
    \]
\end{theorem}

\begin{proof}
    Let \( x \in M \) with \( w_k x \le 0 \). Then
    \[
        w^*w_{k + 1} = w^*w_k + w^*x \ge w^*w_k + \delta
    \]
    Furthermore
    \[
        \|w_{k + 1}\|^2 = \|w_k + x\|^2 = \|w_k\|^2 + 2w_k x + \|x\|^2 \le \|w_k\|^2 + r^2
    \]

    The same estimations hold for \( x \in N \) with \( w_k x \ge 0 \) and using induction yields
    \[
        w^*w_k \ge k\delta \text{ and } \|w_k\|^2 \le kr^2
    \]
    Assuming \( \|w^*\| = 1\) now gives
    \[
        k^2\delta^2 \le \|w_k\|^2 \le kr^2
    \]
    which proves the initial inequality.
\end{proof}
\bigskip


\begin{algorithm}[Perceptron]\label{algo:perceptron}
\end{algorithm}
\inputminted[fontsize=\small, framesep=0.35cm, frame=lines, python3=true]{python}{python/perceptron.py}
\bigskip


\begin{remark}
    Check \hyperref[thm:banach_fix_point]{Banach Fixed-Point Theorem~\ref*{thm:banach_fix_point}} 
    for an alternative proof?!
\end{remark}
\bigskip


\subsection{Feedforward Neural Networks}

\begin{definition}[Activation Functions]
    \hfill
    \begin{enumerate}
        \item The \emph{Heaviside} function \( H: \R \to \{ 0, 1 \} \) is defined as
              \[
                  H(x) = \left \{
                  \begin{array}{ll}
                      1 & \text{ for } x > 0  \\
                      0 & \text{ for }x \le 0
                  \end{array}
                  \right.
              \]
        \item The \emph{sigmoid} function \( \sigmoid \in C^\infty(\R) \) is defined as
              \[
                  \sigmoid(x) = \frac{1}{1 + e^{-x}}
              \]
        \item A function \( s: \R \to \R \) with \( s(x) \to 0 \) 
              for \( x \to -\infty \) and \( s(x) \to 1 \) for \( x \to \infty \) is called a 
              \emph{sigmoidal activation function}
      \end{enumerate}
\end{definition}
\bigskip


\begin{remarks}
    \hfill
    \begin{enumerate}
        \item Since the heaviside function is not continuous and therefore not differentiable 
            at \( 0 \) the sigmoid function can be considered its smooth counterpart
        \item The definition of the sigmoid function yields \( 0 < \sigmoid(x) < 1 \) as well as
              \( \sigmoid(x) \to 0 \) for \( x \to -\infty \) and \( \sigmoid(x) \to 1 \) 
              for \( x \to \infty \)
        \item The quotient rule yields
              \[
                  \sigmoid'(x)
                  = -\frac{-e^{-x}}{{(1 + e^{-x})}^2}
                  = \sigmoid(x) \frac{1 + e^{-x} - 1}{1 + e^{-x}}
                  = \sigmoid(x)(1 - \sigmoid(x))
              \]
              and \( \sigmoid \) is monotonically increasing over its domain
    \end{enumerate}
\end{remarks}
\bigskip


\begin{figure}[H]
    \centering
    \plotsigmoid{}
    \caption{The sigmoid function \( \sigmoid(x) = \frac{1}{1 + e^{-x}} \)}\label{fig:sigmoid}
\end{figure}
\bigskip


\begin{definition}[Single Layer Neural Network]
    Let \( w_k \in \R^n \) and \( b_k \in \R \). For \( \alpha_k \in \R \) define \( f : \Rn \to \R \)
    \[
        f(x) = \sum_{k = 0}^m \alpha_k \sigmoid(w_k x + b_k)
    \]
    The \( f \) is called a \emph{feedforward neural network} of \( m \) units.
\end{definition}
\bigskip

\begin{theorem}[The Universal Approximation Theorem]\label{thm:universal_approximation}
    The subspace of all single layer feedforward neural networks is dense in \( C(I^n) \).
\end{theorem}


\begin{remarks}
    \hfill
    \begin{enumerate}
        \item Let \( f = (f_1, f_2, \dots f_n) : \Rn \to \Rm \) and 
            \[ 
                g(x) = \|f(x)\|^2 = \sum_{j = 1}^m {f_j(x)}^2 
            \] 
        Then
            \[ 
                \frac{\partial g_j}{\partial x_i}(x)
                    = \sum_{j = 1}^m \frac{\partial f_j^2}{\partial x_i}(x)
                    = 2 \sum_{j = 1}^m \frac{\partial f_j}{\partial x_i}(x) f_j(x)
            \] 
        Hence
        \[ 
            \gradient g(x) = 2 \sum_{j = 1}^m \gradient f_j(x) f_j(x)
        \] 

    \end{enumerate}
\end{remarks}
\bigskip


\begin{definition}[Error Function]
    Let \( w \in \R^{n \times m} \) and \( b \in \Rm \). For a finite \emph{test set} \( T \subset \Rn \times \Rm \) 
    define the \emph{error function}
    \[
        E(w,b) = \sum_{(x,y) \in T} \|\sigmoid(wx + b )- y\|^2
    \]
\end{definition}
\bigskip

\subsection{The Backpropagation Algorithm}
