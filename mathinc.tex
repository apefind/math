%%
%% FILE INCLUDE.TEX
%%

%%
%% VERAENDERTE VARIABLEN
%%

\renewcommand{\epsilon}{\varepsilon}
\renewcommand{\phi}{\varphi\,}
\renewcommand{\Re}{{\rm Re}\,}
\renewcommand{\Im}{{\rm Im}\,}




%%
%% NEUE VARIABLEN
%%

\newcommand{\N}{{\bf N}}
\newcommand{\Z}{{\bf Z}}
\newcommand{\Q}{{\bf Q}}
\newcommand{\R}{{\bf R}}
\newcommand{\C}{{\bf C}}
\newcommand{\K}{{\bf K}}

\newcommand{\id}{{\rm id}}
\newcommand{\dist}{{\rm dist}}
\newcommand{\supp}{{\rm supp}}
\newcommand{\disc}{{\bf D}}

\newcommand{\spec}{\sigma}
\newcommand{\resol}{\rho}
\newcommand{\annihil}{\perp}
\newcommand{\isomorph}{\cong}
\newcommand{\tensor}{\otimes}
\newcommand{\directsum}{\oplus}
\newcommand{\inclusion}{\hookrightarrow}

\newcommand{\with}{\;|\;}
\newcommand{\ldot}{\,.\,}
\newcommand{\epstensor}{\,\epsilon\,}
\newcommand{\itensor}{\,\hat{\tensor}_\epsilon}
\newcommand{\ptensor}{\,\hat{\tensor}_\pi}
\newcommand{\compact}{\subset\subset}
\newcommand{\sphere}{\widehat{\C}}

\newcommand{\rd}[1]{{\partial #1}}
\newcommand{\cl}[1]{\overline{#1}}
\newcommand{\oh}[1]{\stackrel{\circ}{#1}}
\newcommand{\polar}[1]{{#1^\circ}}
\newcommand{\conj}[1]{\bar{#1}}
\newcommand{\text}[1]{\mbox{#1}}




%%
%% VERAENDERTE UMGEBUNGEN
%%

\renewcommand{\(}{\begin{equation}}
\renewcommand{\)}{\end{equation}}




%%
%% NEUE UMGEBUNGEN
%%

\newcommand{\sectionA}[2]{%
\refstepcounter{subsection}%
\ifthenelse{\equal{#2}{}}{}{\label{#1-#2}}%
\bigskip\smallskip%
{\bf{\thesubsection} #1.}%
\sl}

\newcommand{\sectionB}[2]{%
\refstepcounter{subsection}%
\ifthenelse{\equal{#2}{}}{}{\label{#1-#2}}%
\bigskip\smallskip%
{\bf{\thesubsection} #1.}%
\rm}

\newcommand{\Unterabschnitt}[2]{%
\refstepcounter{subsection}%
\ifthenelse{\equal{#2}{}}{}{\label{Unterabschnitt-#2}}%
\bigskip\smallskip%
{\bf{\thesubsection} #1.}%
\rm}

\newcommand{\Satz}[1]{\sectionA{Satz}{#1}}
\newcommand{\Lemma}[1]{\sectionA{Lemma}{#1}}
\newcommand{\Folgerung}[1]{\sectionA{Folgerung}{#1}}
\newcommand{\Folgerungen}[1]{\sectionA{Folgerungen}{#1}}

\newcommand{\Definition}[1]{\sectionB{Definition}{#1}}
\newcommand{\Definitionen}[1]{\sectionB{Definitionen}{#1}}
\newcommand{\Bemerkung}[1]{\sectionB{Bemerkung}{#1}}
\newcommand{\Bemerkungen}[1]{\sectionB{Bemerkungen}{#1}}
\newcommand{\Beispiel}[1]{\sectionB{Beispiel}{#1}}
\newcommand{\Beispiele}[1]{\sectionB{Beispiele}{#1}}
\newcommand{\Bezeichnung}[1]{\sectionB{Bezeichnung}{#1}}
\newcommand{\Bezeichnungen}[1]{\sectionB{Bezeichnungen}{#1}}

\newcommand{\Beweis}{\smallskip{\rm\bf Beweis. }\rm}
\newcommand{\BeweisEnde}{\hfill{\Large\rule{1ex}{1ex}}}
\newcommand{\Text}{\bigskip\rm}




%%
%%  NEUE LISTEN, LISTEN�NDERUNGEN
%%


\renewcommand{\labelenumi}{(\alph{enumi})}
\renewcommand{\labelenumii}{\arabic{enumii}.}

\newcounter{listcounter}
\newenvironment{properties}{%
\begin{list}%
{\arabic{listcounter}.}{\usecounter{listcounter}}}%
{\end{list}}
